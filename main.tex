% Author - Jon Arnt Kårstad, NTNU IMT
\documentclass{article}

% Importing document settings from our file "packages.sty"
\usepackage{packages}

% Beginning of document
\begin{document}

% Inserting title page
\import{./}{title}

% Defining front matter settings (Norsk: innstillinger for forord m.m.)
\frontmatter

% Inserting table of contents
\tableofcontents

% Inserting list of figures & list of tables
\listoffigures
\listoftables

% Defining main matter settings (Norsk: innstillinger for hoveddelen av teksten)
\mainmatter

% Introduction explaining this LaTeX-template
%% \import{./Sections/}{}



% Example section added directly into the main-file

\textbf{Forretningside:}
Minimere bruk av strøm ved hjelp av AI


% Printing bibliography
\newpage
\printbibliography[heading = bibintoc, title = Bibliography]    % 'bibintoc' inserts our bibliography into the table of contents


\section{Ide}
-Bygg og eiendomsforvaltning (Smart Building/Facility Management):

-Energistyring og optimalisering

Problem/Behov: Bygninger står for en stor del av energiforbruket i samfunnet, og ineffektiv styring av varme, kjøling og ventilasjon fører til unødvendig høye kostnader og klimagassutslipp.
AI-løsning: Bruk av sensorer og dataanalyse for å justere lys, varme og ventilasjon i sanntid basert på belegg, vær, tidspunkt på døgnet osv.
Verdiskaping: Potensielt store kostnadsbesparelser, redusert energiforbruk og bedre inneklima

% Inserting appendix with separate settings

\section{Innledning (maksimalt 250 ord)}
\subsection*{Relevans}
Hvorfor er det viktig å se nærmere på det dere skriver om i rapporten?

\subsection*{Problem og behov}
Kort om det underliggende problemet/behovet dere har valgt å ta utgangspunkt i, og hvordan deres konsept (produkt/tjeneste) kan løse dette problemet.

\subsection*{Oppsummering av hovedfunn}
Innledningen skal inneholde en kort oppsummering av hovedfunn, og av evalueringen av konseptet fra bærekraftsmessig, samfunnsmessig og økonomisk ståsted.

\section{Problem og utvikling av teknologisk konsept}
\subsection*{Problem/behov}
Beskrivelse av problemet/behovet i valgt bransje.

\subsection*{Teknologisk konsept}
Beskrivelse av konseptet med kommersielt potensial.

\subsection*{Samfunnsmessige og bærekraftsmessige utfordringer}
Vurdering av om problemet er økende frem mot 2030, og hvordan konseptet støtter nødvendig omstilling.

\subsection{Teknologi}
% \begin{enumerate}
%     \item Sensorer å bruke:
%     \begin{itemize}
%         \item Temperatur
%         \item Fuktighet
%         \item CO$_2$
%         \item Luftkvalitet
%     \end{itemize}
%     \item Data innhenting:
%     \begin{itemize}
%         \item Værdata
%         \item Er lokalet booket, hvor lenge
%         \item Hvor mange er i lokalet
%     \end{itemize}
%     \item Arkitektur
%     \begin{itemize}
%         \item Modulær arkitektur
%         \item Reactive AI
%     \end{itemize}
% \end{enumerate}


% Tekst, kan bli for mye
Bruken av sensorer og data i et moderne bygg klima er essensielt for å produsere et behagelig, energi-effektivt og bærekraftig inneklima. De følgende metodene sikrer at framtidige bygginger opprettholder dette.

\subsubsection{Sensorer}\label{section:sensors}
Temperatursensorer spiller en sentral rolle i overvåkningen og reguleringen av et innendørs klimaanlegg. Ved å bruke sanntids data gjør slike sensorer det mulig for klimaanlegget å opprettholde en jevn og behagelig temperatur i byggets lokaler. Dette sikrer at lokalet ikke blir for varmt eller kaldt, noe som bidrar til energieffektivitet og komfort i lokalet.
\\
\\
Fuktighets sensorer er viktige, de overvåker fuktigheten i lokalene. Å opprettholde en fuktighet på 20\% - 40\% er avgjørende for å forhindre ubehag forårsaket  av enten for tørr eller fuktig luft. I tillegg bidrar ideell fuktighet i å ivareta bygget og møbler og hindrer oppbygging av mugg. 
\\
\\
Karbondioksid (CO$_2$) sensorer brukes til å overvåke luftkvaliteten i luften, høy verdi av CO$_2$ er ofte et tegn på dårlig ventilasjon og spesielt i rom med mange personer. Dette kan også være en indikasjon på at det er mange i et rom. Utover generelt luftkvalitets målinger er viktige sensorer, slike sensorer måler etter \textit{PM2.5} altså partikler som er mindre enn 2.5 \micro \hspace{1mm}meter, karbonmonoksid(CO), ozon (O3), nitrogendioksid (NO2), svoveldioksid (SO2) og flyktige organiske forbindelser (VOC). Dette er gasser som kan være farlige for mennesker. 


\subsubsection{Data innhenting}
Innhenting av data er en vital del av regulere ventilasjonen av byggingen. Dette gir nødvendig informasjon til kunstig intelligens (KI) modellen for å ta besluttinger. Det er flere typer av data som er nødvendig til at modellen skal fungere.

Værdata gir informasjon om ytre påvirkninger til lokasjonen. Dette kan enkelt hentes inn ved å bruke \textit{API} til nettsider som \href{https://developer.yr.no/}{\textit{Yr}} og \href{https://api.windy.com/point-forecast/docs#parameters}{\textit{Windy}} her kan data som luftfuktighet, vind hastighet og temperatur gi viktig informasjon til KI modellen. Samtidig vil informasjon om rom booking gi informasjon om lokalet vil være i bruk i framtiden og okkupasjons data gi sanntid data om hvor mange som er i lokalet. Denne dataen ilag med \textit{IOT} fra sensorene nevnt i avsnittet over \ref{section:sensors} fullføre dataen KI modellen vil trenge for å regulere lokalene. 



\subsubsection{Arkitektur}
Applikasjonen vil følge en modulær arkitektur 
% Kansje vurdere meir beskrivende namn
\subsubsection{KI}
Kunstig intelligens (KI) ligger sentralt i å kunne bruke software for å effektivisere bygget. Produktet bruker et smalt KI

\subsection*{Gruppens styrker}
Beskrivelse av styrker som gjør gruppen godt egnet til å løse problemet.

\section{Mulighetsanalyse}
\subsection*{Markedspotensial}
\begin{itemize}

% Nicolay: 1,2,3,4, 9
% Jo: 5, 6, 7, 8, 10

    %  1, Nicolay,  Hvem er kunde og hvem er bruker av konseptet deres? Er dette samme beslutningstaker?
    \item \textbf{Hvem er kunde og bruker?} 

    \begin{itemize}
        % \item \textbf{Kunde:} Eiendomsforvaltere, bedriftsledere og energiansvarlige i kommersielle bygg.
        \item Kunden for vårt produkt er hovedsakelig eiendomsforvaltere, bedriftslederer og energiansvarlige i kommersielle bygg. Disse aktørene har ansvar for drift, vedlikehold og økonomisk styring i byggene. De er på jakt etter løsninger som reduserer driftsoknstandene og forbedrer energieffektiviteten.
        
        
        % \item \textbf{Bruker:} Ansatte og leietakere som opplever fordelene ved optimalisert energiforbruk.
        \item Brukerne av produktet vil være ansatte og besøkene leietagere som bruker energi i bygget. Ved at brukerne. Ai-modellen samler inn data basert på brukerens vaner og rutiner og analyserer daglig bruk av belysning, temeraturstyring og annen energibruk 
        \item \textbf{Beslutningstaker:} Typisk eiendomsforvaltere eller finansansvarlige, med mulig involvering av bærekraftsansvarlige.
    \end{itemize}

    % 2, Nicolay, Hvordan løser kunden/brukeren problemet (dekker behovet) i dag?
    \item \textbf{Hvordan løser kunden problemet i dag?} 
    Dagens systemer er ofte basert på enkel tidsstyring, noe som fører til ineffektiv bruk av energi.

    % 3, Nicolay Hvordan skaper konseptet deres verdi? 
    \item \textbf{Verdi som konseptet skaper.} 
    \begin{itemize}
        \item Reduserte driftskostnader gjennom effektiv energiutnyttelse.
        \item Bidrag til oppnåelse av bærekraftsmål og grønnere drift.
        \item Økt brukertilfredshet ved bedre komfort og ressursutnyttelse.
    \end{itemize}

    % 4, Nicolay Bestem pris for konseptet, og beskriv hvordan dere fastsetter prisen og hva dere baserer vurderingen på. Er kundene villige til å betale denne prisen? Benytt kundeundersøkelser eller andre kilder i argumentasjonen.
    \item \textbf{Prisfastsettelse og betalingsvillighet.} %2
    \begin{itemize}
        \item Abonnementsmodell, for eksempel en månedlig avgift basert på kvadratmeter.
        \item Engangskostnad for installasjon, avhengig av byggets størrelse og sensorbehov.
        \item Kundetilbakemeldinger og markedsundersøkelser kan brukes til å bekrefte betalingsvillighet.
    \end{itemize}
    
    % 5, Jo, Kan man segmentere markedet i ulike delmarked? Beskriv markedet/markedssegmentet dere vil fokusere på og estimer markedsstørrelsen – hvor mange potensielle kunder eksisterer? Ta forutsetninger og begrunn de der det er nødvendig.
    \item \textbf{Markedssegmentering og estimering av markedsstørrelse.} 
    \begin{itemize}
        \item \textbf{Primære segmenter:} Større kommersielle bygg som kontorer, kjøpesentre og hoteller.
        \item \textbf{Sekundære segmenter:} Mindre bedrifter og offentlige bygg som skoler og sykehus.
        \item \textbf{Geografisk fokus:} Start i Norge med mulighet for utvidelse til internasjonale markeder.
    \end{itemize} 

     % 6, Jo, Bruk kilder, argumenter for og tallfest hvordan dere estimerer at markedet vil vokse fremover. 
    \item \textbf{Vekstprognoser for markedet.} 
    \begin{itemize}    
        \item Estimer antall kommersielle bygg i målområdet (f.eks. Norge har over 500 000 næringsbygg).
        \item Beregn potensial ved å multiplisere antall bygg med gjennomsnittlig kostnad per bygg for systemet.
        \item Forventet vekst i markedet, drevet av strenge bærekraftskrav og økt energikostnad.
    \end{itemize}
\end{itemize}

% 7, Jo, Forretningsmodell, kommersialisering og økonomiske betraktninger
% Benytt Business Model Canvas og fyll ut forretningsmodellen for konseptet deres. 
\subsection*{Forretningsmodell og kommersialisering}
\textbf{Business Model Canvas:} 
\begin{itemize}
    \item \textbf{Kjerneaktiviteter:} Systemutvikling, modelltrening, installasjon og vedlikehold.
    \item \textbf{Verdiforslag:} Reduksjon av energiforbruk og kostnader, samtidig som bærekraftsmål oppfylles.
    \item \textbf{Kunderelasjoner:} Direkte salg og vedlikeholdsavtaler.
    \item \textbf{Distribusjonskanaler:} Salgsteam og partnere innen eiendomsdrift.
    \item \textbf{Inntektsstrømmer:} Abonnementstjenester og engangskostnader for implementasjon.
    \item \textbf{Nøkkelressurser:} Sensorteknologi, skyplattformer og KI-utviklingsteam.
    \item \textbf{Partnere:} Leverandører av sensorer og automasjonssystemer.
    \item \textbf{Kostnadsstruktur:} Utviklingskostnader, sensorer og markedsføring.
\end{itemize}

% 8, Jo, Beskriv hvordan dere planlegger å kommersialisere konseptet, og hva de viktigste ressursene som trengs til dette er
\subsubsection*{Kommersialisering}
\begin{itemize}
    \item Start med pilotprosjekter for å demonstrere løsningens verdi.
    \item Tilby skalerbare løsninger tilpasset ulike byggtyper og størrelser.
\end{itemize}


% 9, Nicolay, Beskriv hva de viktigste kostnadene er, og hvor de viktigste inntektene kommer fra.


\subsubsection*{Kostnader og finansiering}
\begin{itemize}
    \item \textbf{Hovedkostnader:} KI-utvikling, sensorer og skyplattform.
    \item \textbf{Inntektskilder:} Abonnementer og engangssalg.

    % 10, Jo,  Hvilke finansieringskilder er relevante for konseptet, hvorfor?
    \item \textbf{Finansiering:} Tilskudd fra Innovasjon Norge, støtteordninger for grønne løsninger, og private investorer.
\end{itemize}


\section{Miljøanalyse og bærekraftsperspektiver}
Ifølge et blogginnlegg fra GreenMatch, et bærekraftig energiselskap fra Storbritannia, går 34\% av energien som når sluttbrukeren til spille. Dette kan skyldes ineffektive apparater, men også ineffektive energivaner, som å la innretninger stå på når de ikke er i bruk. Denne energisløsingen fører direkte til klimaeffekter, ikke bare som konsekvenser av den bortkastede energien, men også i prosessene som kreves for å produsere energien som til slutt går til spille. Ifølge den Europeiske Unionen (EU) har energitilgjengeligheten gått ned de siste årene i lag med tilgjengelighet på fossile brennstoff, noe som gjør det viktigere enn noen gang å utnytte energien vi har på en bedre måte. Ved å redusere energisløsing hos sluttbrukeren, reduseres sluttbrukerens energiforbruk, energietterspørselen senkes, og med nok innsats kan forurensning fra energiproduksjon reduseres drastisk.

Boreas har som mål å øke energieffektiviteten hos sluttbrukere ved å redusere energisløsing som oppstår på grunn av ineffektive energivaner. Ved å automatisere strømstyring på kontorarealer ved hjelp av KI, kan Boreas fremme energieffektive praksiser basert på bruksmønstre. Den kan slå av lys etter arbeidstid, sørge for at varmeovner kun brukes til å oppnå stabile temperaturer, og kontrollere mange andre apparater med hensyn til eksterne faktorer som andre automatiseringstjenester sjeldent tar i bruk. Ved å implementere Boreas på kontorplassen kan selskaper redusere strømregninger, øke komforten på arbeidsplassen, og bidra til en konkret positiv innvirkning på miljøet.

Ved å implementere Boreas kan selskaper tilpasse seg FNs bærekraftsmål (UN SDG) samt andre lignende mål i forskjellige verdensregioner. Boreas reduserer energiforbruket og øker andelen energi som brukes til meningsfulle formål, noe som er i tråd med SDG 12 (Bærekraftig forbruk), ulike energieffektivitetslover og -programmer i USA, samt EUs Green Deal-politikk. Ved å redusere energisløsing bidrar selskaper også indirekte til å redusere sine utslipp og fremme et grønnere klima i samsvar med SDG 13 (Klimatiltak).

\subsection*{Produksjon, bruk og avhending}
Betraktninger rundt energi, klima, ressursbruk, helse og økosystemer.

\subsection*{Miljøvurderingsmetoder}
Beskrivelse av metoder som livsløpsvurdering, fotavtrykksanalyse eller materialstrømsanalyse.

\section{Sammenstilling, refleksjon og diskusjon (maksimalt 750 ord)}
\subsection*{Kritiske refleksjoner}
Problemstillinger som har oppstått ved overgang til bruk av kunstig intelligens.

\subsection*{Mål for konseptet}
Hva som legges til grunn som mål.

\subsection*{Bærekraft og avveininger}
Diskusjon av økonomisk, samfunnsmessig og miljømessig bærekraft.

\subsection*{Usikkerhetsmomenter}
Refleksjon over begrensninger og usikkerhetsmomenter.

\subsection*{Veien videre}
Er konseptet verdt å gå videre med?

\section{Referanseliste}
\begin{itemize}
    \item [1.] \emph{Referanse 1}
    \item [2.] \emph{Referanse 2}
    \item [3.] \emph{Referanse 3}
\end{itemize}

\section{Vedlegg}
\subsection*{Kontaktlogg}
Beskrivelse av kontakt med eksterne aktører.

\subsection*{Beregninger}
Detaljerte utregninger.


\addappendix
\import{./Appendices/}{example_appendix}



% End of document
\end{document}
