\section{Mulighetsanalyse}
\subsection*{Markedspotensial}
\begin{itemize}

% Nicolay: 1,2,3,4, 9
% Jo: 5, 6, 7, 8, 10

    %  1, Nicolay,  Hvem er kunde og hvem er bruker av konseptet deres? Er dette samme beslutningstaker?
    \item \textbf{Hvem er kunde og bruker?} 

    \begin{itemize}
        % \item \textbf{Kunde:} Eiendomsforvaltere, bedriftsledere og energiansvarlige i kommersielle bygg.
        \item Kunden for vårt produkt er hovedsakelig eiendomsforvaltere, bedriftslederer og energiansvarlige i kommersielle bygg. Disse aktørene har ansvar for drift, vedlikehold og økonomisk styring i byggene. De er på jakt etter løsninger som reduserer driftsoknstandene og forbedrer energieffektiviteten.
        
        
        % \item \textbf{Bruker:} Ansatte og leietakere som opplever fordelene ved optimalisert energiforbruk.
        \item Brukerne av produktet vil være ansatte og besøkene leietagere som bruker energi i bygget. Ved at brukerne. Ai-modellen samler inn data basert på brukerens vaner og rutiner og analyserer daglig bruk av belysning, temeraturstyring og annen energibruk 
        \item \textbf{Beslutningstaker:} Typisk eiendomsforvaltere eller finansansvarlige, med mulig involvering av bærekraftsansvarlige.
    \end{itemize}

    % 2, Nicolay, Hvordan løser kunden/brukeren problemet (dekker behovet) i dag?
    \item \textbf{Hvordan løser kunden problemet i dag?} 
    Dagens systemer er ofte basert på enkel tidsstyring, noe som fører til ineffektiv bruk av energi.

    % 3, Nicolay Hvordan skaper konseptet deres verdi? 
    \item \textbf{Verdi som konseptet skaper.} 
    \begin{itemize}
        \item Reduserte driftskostnader gjennom effektiv energiutnyttelse.
        \item Bidrag til oppnåelse av bærekraftsmål og grønnere drift.
        \item Økt brukertilfredshet ved bedre komfort og ressursutnyttelse.
    \end{itemize}

    % 4, Nicolay Bestem pris for konseptet, og beskriv hvordan dere fastsetter prisen og hva dere baserer vurderingen på. Er kundene villige til å betale denne prisen? Benytt kundeundersøkelser eller andre kilder i argumentasjonen.
    \item \textbf{Prisfastsettelse og betalingsvillighet.} %2
    \begin{itemize}
        \item Abonnementsmodell, for eksempel en månedlig avgift basert på kvadratmeter.
        \item Engangskostnad for installasjon, avhengig av byggets størrelse og sensorbehov.
        \item Kundetilbakemeldinger og markedsundersøkelser kan brukes til å bekrefte betalingsvillighet.
    \end{itemize}
    
    % 5, Jo, Kan man segmentere markedet i ulike delmarked? Beskriv markedet/markedssegmentet dere vil fokusere på og estimer markedsstørrelsen – hvor mange potensielle kunder eksisterer? Ta forutsetninger og begrunn de der det er nødvendig.
    \item \textbf{Markedssegmentering og estimering av markedsstørrelse.} 
    \begin{itemize}
        \item \textbf{Primære segmenter:} Større kommersielle bygg som kontorer, kjøpesentre og hoteller.
        \item \textbf{Sekundære segmenter:} Mindre bedrifter og offentlige bygg som skoler og sykehus.
        \item \textbf{Geografisk fokus:} Start i Norge med mulighet for utvidelse til internasjonale markeder.
    \end{itemize} 

     % 6, Jo, Bruk kilder, argumenter for og tallfest hvordan dere estimerer at markedet vil vokse fremover. 
    \item \textbf{Vekstprognoser for markedet.} 
    \begin{itemize}    
        \item Estimer antall kommersielle bygg i målområdet (f.eks. Norge har over 500 000 næringsbygg).
        \item Beregn potensial ved å multiplisere antall bygg med gjennomsnittlig kostnad per bygg for systemet.
        \item Forventet vekst i markedet, drevet av strenge bærekraftskrav og økt energikostnad.
    \end{itemize}
\end{itemize}

% 7, Jo, Forretningsmodell, kommersialisering og økonomiske betraktninger
% Benytt Business Model Canvas og fyll ut forretningsmodellen for konseptet deres. 
\subsection*{Forretningsmodell og kommersialisering}
\textbf{Business Model Canvas:} 
\begin{itemize}
    \item \textbf{Kjerneaktiviteter:} Systemutvikling, modelltrening, installasjon og vedlikehold.
    \item \textbf{Verdiforslag:} Reduksjon av energiforbruk og kostnader, samtidig som bærekraftsmål oppfylles.
    \item \textbf{Kunderelasjoner:} Direkte salg og vedlikeholdsavtaler.
    \item \textbf{Distribusjonskanaler:} Salgsteam og partnere innen eiendomsdrift.
    \item \textbf{Inntektsstrømmer:} Abonnementstjenester og engangskostnader for implementasjon.
    \item \textbf{Nøkkelressurser:} Sensorteknologi, skyplattformer og KI-utviklingsteam.
    \item \textbf{Partnere:} Leverandører av sensorer og automasjonssystemer.
    \item \textbf{Kostnadsstruktur:} Utviklingskostnader, sensorer og markedsføring.
\end{itemize}

% 8, Jo, Beskriv hvordan dere planlegger å kommersialisere konseptet, og hva de viktigste ressursene som trengs til dette er
\subsubsection*{Kommersialisering}
\begin{itemize}
    \item Start med pilotprosjekter for å demonstrere løsningens verdi.
    \item Tilby skalerbare løsninger tilpasset ulike byggtyper og størrelser.
\end{itemize}


% 9, Nicolay, Beskriv hva de viktigste kostnadene er, og hvor de viktigste inntektene kommer fra.


\subsubsection*{Kostnader og finansiering}
\begin{itemize}
    \item \textbf{Hovedkostnader:} KI-utvikling, sensorer og skyplattform.
    \item \textbf{Inntektskilder:} Abonnementer og engangssalg.

    % 10, Jo,  Hvilke finansieringskilder er relevante for konseptet, hvorfor?
    \item \textbf{Finansiering:} Tilskudd fra Innovasjon Norge, støtteordninger for grønne løsninger, og private investorer.
\end{itemize}